\documentclass[xcolor=table]{beamer}
\usetheme{Boadilla}

% ------ Basic packages -------
%\usepackage[margin=1.2in]{geometry}	% smaller margins              
\usepackage[utf8]{inputenc}			%for æ,ø,å on mac
\usepackage{amsmath,amsfonts,amssymb,amsthm,mathtools} % for mathematics
\usepackage{hyperref}					   % Clickable links to urls, internal and external references etc.
\usepackage{pdfpages}					 % for å hente in pdf
\usepackage{cite}								% For sitater og referanser
\usepackage[square,sort]{natbib}	% Referansestil  

% ------ Figures/graphic -----
\usepackage{graphicx}					   %Graphic package
\usepackage[table]{xcolor}				 % For coloured rows or columns in tables
\usepackage[font=small,labelfont=bf,textfont=normal, format=hang]{caption}	% Configuration of captions
\usepackage{subcaption,graphicx}
\usepackage{verbatim,longtable}		% For lange tabeller over flere sider
\usepackage{wrapfig}  						% Wrap text around figures	
\usepackage{float} 								% To place figures correctly
\usepackage{color}							   % muliggjør farget tekst
\usepackage{tcolorbox}					   % enables color boxes
\definecolor{light-gray}{gray}{0.95} %the shade of grey that stack exchange uses
\usepackage{multirow}						
\usepackage{bm}									% usepackage bold symbols
\usepackage{xspace}							 % trengs også for fete typer

% -------- TikZ --------
\usepackage{tikz}
\usetikzlibrary{shapes.geometric, arrows}
\usetikzlibrary{matrix,chains,positioning,decorations.pathreplacing,arrows}
\tikzstyle{startstop} = [rectangle, rounded corners, minimum width=1.8cm, minimum height=1cm,text centered, draw=black, fill=blue!40]
\tikzstyle{io} = [trapezium, trapezium left angle=80, trapezium right angle=100, minimum width=1cm, minimum height=1cm, text centered, draw=black, fill=blue!20]
\tikzstyle{process} = [rectangle, minimum width=2cm, minimum height=1cm, text centered, text width=2cm, draw=black, fill=blue!5]
\tikzstyle{decision} = [diamond, minimum width=2cm, minimum height=1cm, text centered, draw=black, fill=green!30]
\tikzstyle{arrow} = [thick,->,>=stealth]

% -------- Embedded code ----------
\usepackage{listings}
\usepackage{color}
\definecolor{dkgreen}{rgb}{0,0.6,0}
\definecolor{gray}{rgb}{0.5,0.5,0.5}
\definecolor{mauve}{rgb}{0.58,0,0.82}
\lstset{frame=tb,
	language=Bash,
	aboveskip=3mm,
	belowskip=3mm,
	showstringspaces=false,
	columns=flexible,
	basicstyle={\small\ttfamily},
	numbers=none,
	numberstyle=\tiny\color{gray},
	keywordstyle=\color{blue},
	commentstyle=\color{dkgreen},
	stringstyle=\color{mauve},
	breaklines=true,
	breakatwhitespace=true,
	tabsize=4
}

% ------ Utseende og forside -----
\title{WRF step-by-step}
\subtitle{Testcase at Nygårdsfjellet}
\author{Torgeir Blæsterdalen}
\institute{Department of Industrial Engineering, UiT-The Arctic University of Norway}
\date{\today}


\begin{document}
\begin{frame}
\titlepage
\end{frame}


\begin{frame}{Lecture topics}
\begin{enumerate}
	\item[1] Test-case on Stallo
		\begin{itemize}
			\item[1.1] Terrestrial data download
			\item[1.2] Meteorological input data
			\item[1.3] WPS
			\item[1.4] WRF
		\end{itemize}
	\item[2] Streamlining steps into a jobscript
	\item[3] Inspecting simulation variables
	\item[4] Summary of WRF configurations
	\item[5] Some useful Unix/Linux commands
	\item[6] Useful links
\end{enumerate}
\end{frame}


\begin{frame}{1 Test-case on Stallo}
\framesubtitle{1.1 Terrestrial input data}
 Terrestrial data: Downloaded from the National Centre for Atmospheric Reseach (NCAR): \url{http://www2.mmm.ucar.edu/wrf/users/download/get_sources_wps_geog.html}
 
 \vskip 1 cm
 The complete  terrestrial dataset is large and it might be a good idea not to download, but to create a link to already existing terrestrial files on Stallo.  
\end{frame}

\begin{frame}[fragile, allowframebreaks=1, t]{1.2 Meteorological input data}
Meteorological boundary conditions: In this case, ERA-Interim is used. Can also use f.ex. on-site measurements or remote sensing dataproducts. The ERA-Interim data used in this case has a temporal and spatial resolution of 6 hours and approximately 80 km at 60 vertical levels. 
Procedure for downloading ERA-Interim data from the Research Data Archive. The user has to log in and register for data access. 
The meteorological boudary conditions consists of both pressure files and surface variables. 
\begin{enumerate}
	\item Go to \url{http://rda.ucar.edu/datasets/ds627.0/}
	\begin{itemize}
		\item Click data access and scroll to \textit{ERA Interim atmospheric model analysis interpolated to pressure levels}. Click \textit{Web File Listing}.
		\item Select \textit{Faceted Browse}.
	\end{itemize}
	\item
	\begin{itemize}
		\item  Select start and end date and time and select \textit{continue}.
		\item Choose \textit{on} for \textit{range selection}.
		\item Select all \texttt{ei.oper}-variables
		\item Click on create a unix file using \texttt{Wget}. 
	\end{itemize}
	The script will open in a new window.

	\item Log in to Stallo and onto a computational node
		\begin{lstlisting}[backgroundcolor = \color{light-gray}, language=bash]
		ssh -X <username>@stallo.uit.no
		srun --nodes=1 --ntasks-per-node=1 --time=02:00:00 --pty bash -i
		\end{lstlisting}
	\item Make a directory for the input met data and move into the directory
		\begin{lstlisting}[backgroundcolor = \color{light-gray}, language=bash]
		mkdir <directory name>
		cd <directory name>
		\end{lstlisting}
	\item Create a empty file
		\begin{lstlisting}[backgroundcolor = \color{light-gray}, language=bash]
		vi PressureDownload.sh 
		\end{lstlisting}
		and copy-paste the generated unix-script in the empty file. 
	\item Change the password to the one used at UCAR (line 19). Write the changes to the file and quit, hitting \texttt{esc}, then \texttt{:wq} (write and quit)
	\item Make it executable
		\begin{lstlisting}[backgroundcolor = \color{light-gray}, language=bash]
		chmod 777 PressureDownload.sh
		\end{lstlisting}
	\item Run the file
	\begin{lstlisting}[backgroundcolor = \color{light-gray}, language=bash]
	./PressureDownload.sh
	\end{lstlisting}
	Note: If the download does not work, try changing \texttt{Wget} to \texttt{cURL}.
\end{enumerate}
Re-run step 1-7, but select \textit{ERA Interim atmospheric model analysis for surface}
After running the file, all meteorological input files should be created and in the same directory. 
\end{frame}




\begin{frame}[fragile, allowframebreaks=.95, t]{1.3 WPS}
Log in to Stallo.
\begin{enumerate}
	\item Move into the \texttt{WPS} folder and load the module required to run WPS
	\begin{lstlisting}[backgroundcolor = \color{light-gray}, language=bash]
	module load WPS/3.9.1-intel-2017a-dmpar
	\end{lstlisting}
	\item Edit the wps namelist
	\begin{lstlisting}[backgroundcolor = \color{light-gray}, language=bash]
	vi namelist.wps
	\end{lstlisting}
	\item Run geogrid to create static data for the domain
	\begin{lstlisting}[backgroundcolor = \color{light-gray}, language=bash]
	geogrid.exe
	\end{lstlisting}
	The geogrid should produce the message
	\begin{lstlisting}[backgroundcolor = \color{light-gray}, language=bash]
	!!!!!!!!!!!!!!!!!!!!!!!!!!!!!!!!!!!!!!!!!!!!!
	!  Successful completion of geogrid.             !
	!!!!!!!!!!!!!!!!!!!!!!!!!!!!!!!!!!!!!!!!!!!!!
	\end{lstlisting}
	This should create the static file \texttt{geo\_em.d01.nc} (\texttt{geo\_em.d02.nc} for two domains, and similar for three). 
	\item Ungrib the data 
	\begin{itemize}
	\item Make a soft-link to the data description file (Vtable)
	\begin{lstlisting}[backgroundcolor = \color{light-gray}, language=bash]
	ln -sf ungrib/Variable_Tables/Vtable.ERA-interim.pl Vtable
	\end{lstlisting}
	\item Make a link to the GRIB data (met-files)
	\begin{lstlisting}[backgroundcolor = \color{light-gray}, language=bash]
	./link_grib.csh /<path to met input>/ei.oper*
	\end{lstlisting}
	If the links are working, they should appear blue. Contrary red, if the files are not found.
	\end{itemize}
	\item Ungrib the meteorological data
	\begin{lstlisting}[backgroundcolor = \color{light-gray}, language=bash]
	ungrib.exe
	\end{lstlisting}
	This run should end with
	\begin{lstlisting}[backgroundcolor = \color{light-gray}, language=bash]
	!!!!!!!!!!!!!!!!!!!!!!!!!!!!!!!!!!!!!!!
	!  Successful completion of ungrib.       !
	!!!!!!!!!!!!!!!!!!!!!!!!!!!!!!!!!!!!!!!
	\end{lstlisting}
	\item Run metgrid
	\begin{lstlisting}[backgroundcolor = \color{light-gray}, language=bash]
	metgrid.exe
	\end{lstlisting}
	As the other WPS executables, this should end with 
	\begin{lstlisting}[backgroundcolor = \color{light-gray}, language=bash]
	!!!!!!!!!!!!!!!!!!!!!!!!!!!!!!!!!!!!!!!
	!  Successful completion of metgrid.      !
	!!!!!!!!!!!!!!!!!!!!!!!!!!!!!!!!!!!!!!!
	\end{lstlisting}
	This step concludes the preprocessing.
\end{enumerate}
\end{frame}


\begin{frame}[fragile]{1.4 WRF}
Move to the WRF directory.
\begin{enumerate}
	\item Create a softlink to the met.-files created in the metgrid-step
	\begin{lstlisting}[backgroundcolor = \color{light-gray}, language=bash]
	ln -sf ../WPS/met_em.d0* .
	\end{lstlisting}
	The \texttt{met\_em.d01*} - files will appear in the directory as links to the met files. Check that the links appear in a cyan-ish color, this indicates a functioning link. Non-functioning links appear red. 
	\item Edit/create namelist.input
	\begin{lstlisting}[backgroundcolor = \color{light-gray}, language=bash]
	vi namelist.input
	\end{lstlisting}
\end{enumerate} 
Because the WRF-step requires parallellization, this is best to do in a jobscript, i.e. not on a interactive node. 
\end{frame}


\begin{frame}[fragile, allowframebreaks=.95, t]{2. Streamlining steps into a jobscript}
\begin{lstlisting}[backgroundcolor = \color{light-gray}, language=bash]
#!/bin/bash
#:---------------------------------------------------------------
# Jobscript for running WRF-simulations in the WRF_narvik folder 
#
# Last edite: 22.March.2018, Torgeir
#:---------------------------------------------------------------

#:---- Estimated resources ----
#SBATCH --job-name=WRFjobscript_testcase
# Stallo account to charge
#SBATCH -A nn9426k
# Computation resources; nodes and cores
#SBATCH --nodes=3
#SBATCH --ntasks-per-node=20
# Runtime: d-hh:mm:ss  (set a bit higher than expected)
#SBATCH --time 0-01:00:00

# Delete all loaded modules (for avoiding conflicts with previously loaded modules)
module purge

#:---- WPS ----
# Move into the WPS test-directory and load modules
cd WPS
module load WPS/3.9.1-intel-2017a-dmpar

# Run geogrid
geogrid.exe

# Make a soft-link to the data secription file (V-table)
ln -sf ungrib/Variable_Tables/Vtable.ERA-interim.pl Vtable

# Make a link to the GRIB data  (input meteorological files)
./link_grib.csh /global/work/blasterdalen/WRF_narvik/MetInput/ei.oper*

# Run ungrib
ungrib.exe

# Run metgrid
metgrid.exe

#:---- WRF ----
# Move to the WRFV3 directory and load WRF model
cd ../WRFV3/
module load WRF/3.9.1-intel-2017a-dmpar

# Make soft-links to the met.-files from the WPS directory
ln -sf ../WPS/met_em.d0* .

# Run real on paralell processes
mpirun -np real.exe

# Run wrf on all of the requested cores
# mpirun -np wrf.exe

exit 0
\end{lstlisting}
\end{frame}




\begin{frame}[fragile]{3. Inspecting output files}
NetCDF files can be examined using 
\begin{lstlisting}[backgroundcolor = \color{light-gray}, language=bash]
ncdump -h met_em.d02.2008-03-04_12:00:00.nc
\end{lstlisting}
Or graphically using the \texttt{ncview} utility. This requires a X-window. For Unix, \texttt{XQuarts} is a good alternative. For Windows, \texttt{Xming}
\begin{lstlisting}[backgroundcolor = \color{light-gray}, language=bash]
module load ncview/2.1.7-intel-2016b
ncview met_em.d02.2008-03-04_12:00:00.nc
\end{lstlisting}
\end{frame}





\begin{frame}[fragile, allowframebreaks=.9, t]{4. Summary of WRF configurations}
\rowcolors{1}{}{light-gray}
\begin{longtable}{ p{4.1cm}  |  p{7.9cm}}
	\hline
	WRF model                           	& WRF V3.9.1                                                                 \\
	WRF dynamical solver           	& ARW                                                                        \\
	Domains                             		& 3$^*$, telescoping nests                                    \\
	2-way nesting                       	& D01: False                                                                 \\
	\rowcolor{light-gray}               & D02: True                                                                  \\
	\rowcolor{light-gray}               & D03: True                                                                  \\
	Domain grid spacing$^*$        & D01: dx = 18000 m, dy = 18000 m                \\
	\rowcolor{white}                    & D02:  dx = 6000 m, dy = 6000 m                         \\
	\rowcolor{white}                    & D03:  dx = 2000 m, dy = 2000 m                        \\
	Map projection                      & Polar stereographic                                                 \\
	Time step                           		& 90 seconds$^*$                                                             \\
	Adaptive time step                  & False                                                                      \\
	Integration scheme                  & RK3                                                                        \\
	Advection scheme                    & 5th order advection                                            \\
	Temporal resolution of out-files    & 10 minutes intervals                                     \\
	Number of vertical levels$^*$       & 51 \\	% , see \figref{HeightCoordinates}.            \\
	Microphysics                        	& WSM5                                                                       \\
	SL scheme                           	& Monin Obhukhov                                                       \\
	Land-surface model                  & Noah Land-Surface model                                 \\
	PBL scheme                          	& MYJ, TKE scheme                                                    \\
	Cumulus                             		& Betts-Miller-Janic scheme                                                  \\
	Number of soil layers               & 4                                                                          \\
	Longwave radiation                  & New Goddard                                                                \\
	Shortwave radiation                 & New Goddard                                                                \\
	Urban physics                       	& Multi-layer BEP scheme (works only with the MYJ and BouLac PBL schemes)    \\
	Grid nudging (Newtonian relaxation) & No grid nudging                                                            \\
	Eddy coefficient option             & Horizontal Smagorinsky 1st order closure (recommended for real-data cases) \\
	Turbulence and mixing               & 2nd order diffusion term (recommended for real-data cases)                 \\
	Base-state sea level temperature    & 290 K  \\
	\hline
\end{longtable}
\end{frame}



\begin{frame}[fragile]{5. Useful Unix/Linux commands}
\begin{tabular}{p{4.5 cm} p{7	 cm}}
	\texttt{cd}										 & Change directory \\
	\texttt{chmod 777 <filename>} 	& Give permission to read, write and execute file/folder	\\
	\texttt{cost} 				  					& View account and core hours 											  \\
	\texttt{du -sh *} 								& List of elements in folder and how large they are \\
	\texttt{ll, ls}										& List of elements in directory \\
	\texttt{ln -sf <path/filename>}		& Create soft-link \\
	\texttt{mkdir}									& Make directory	\\
	\texttt{module avail}					  &  Check of module is available in the Stallo system 			\\	
	\texttt{mv}										& Move file		\\
	\texttt{ncdump -h <filename>}	& View specifications of NetCDF-file 	\\
	\texttt{pwd}									& Present work directory \\
	\texttt{rm, rmdir}							& Remove file or directory, respectively \\
	\texttt{scp}									 & Copy/download from remote to local host \\
	\texttt{squeue -u <username>}	& See status of the submitted job(s)
\end{tabular}
\end{frame}



\begin{frame}{6. Useful links}
\begin{itemize}
	\item WRF user 's guide:\\
	\url{http://www2.mmm.ucar.edu/wrf/users/docs/user_guide_V3.8/contents.html}\\
	\item ARW technical note: (pdf-download)\\
	\url{http://opensky.ucar.edu/islandora/object/technotes:500}
	\item A link to the lecture notes and some uploaded files (like jobscripts and some post-processing scripts) can be viewed and downloaded from my GitHub-account at \\
	\url{https://github.com/torgeirtb?tab=repositories}
\end{itemize}
\end{frame}

\end{document}
\documentclass{beamer}
\usetheme{Boadilla}

% ------ Basic packages -------
%\usepackage[margin=1.2in]{geometry}	% smaller margins              
\usepackage[utf8]{inputenc}			%for æ,ø,å on mac
\usepackage{amsmath,amsfonts,amssymb,amsthm,mathtools} % for mathematics
\usepackage{hyperref}					   % Clickable links to urls, internal and external references etc.
\usepackage{fancyhdr}					  % trengs for topp- og bunntekster og sidetall
\usepackage{pdfpages}					 % for å hente in pdf
%\numberwithin{equation}{section} % sub-numbers for equations
\usepackage{cite}								% For sitater og referanser

% ------ Figures/graphic -----
\usepackage{graphicx}						%Graphic package
\usepackage{xcolor}
\usepackage{multicol}						 % for multiple columns
\usepackage[font=small,labelfont=bf,textfont=normal, format=hang]{caption}	% Configuration of captions
\usepackage{subcaption,graphicx}
\usepackage{wrapfig}  						% Wrap text around figures	
\usepackage{float} 								% To place figures correctly
\usepackage{color}							   % muliggjør farget tekst
\usepackage{tcolorbox}					   % enables color boxes
\definecolor{light-gray}{gray}{0.95} %the shade of grey that stack exchange uses
\usepackage{multirow}						
\usepackage{bm}									% usepackage bold symbols
\usepackage{xspace}							 % trengs også for fete typer
%\usepackage[square,sort]{natbib}				% For referanser i klamme-parenteser  
\usepackage{cite}


% -------- TikZ --------
\usepackage{tikz}
\usetikzlibrary{shapes.geometric, arrows}
\usetikzlibrary{matrix,chains,positioning,decorations.pathreplacing,arrows}
\tikzstyle{startstop} = [rectangle, rounded corners, minimum width=1.8cm, minimum height=1cm,text centered, draw=black, fill=blue!40]
\tikzstyle{io} = [trapezium, trapezium left angle=80, trapezium right angle=100, minimum width=1cm, minimum height=1cm, text centered, draw=black, fill=blue!20]
\tikzstyle{process} = [rectangle, minimum width=2cm, minimum height=1cm, text centered, text width=2cm, draw=black, fill=blue!5]
\tikzstyle{decision} = [diamond, minimum width=2cm, minimum height=1cm, text centered, draw=black, fill=green!30]
\tikzstyle{arrow} = [thick,->,>=stealth]

% -------- Embedded code ----------
\usepackage{listings}
\usepackage{color}
\definecolor{dkgreen}{rgb}{0,0.6,0}
\definecolor{gray}{rgb}{0.5,0.5,0.5}
\definecolor{mauve}{rgb}{0.58,0,0.82}
\lstset{frame=tb,
	language=Matlab,
	aboveskip=3mm,
	belowskip=3mm,
	showstringspaces=false,
	columns=flexible,
	basicstyle={\small\ttfamily},
	numbers=none,
	numberstyle=\tiny\color{gray},
	keywordstyle=\color{blue},
	commentstyle=\color{dkgreen},
	stringstyle=\color{mauve},
	breaklines=true,
	breakatwhitespace=true,
	tabsize=3
}

% Nye kommandoer/snarveier
\newcommand{\pd}{\ensuremath{\partial}} 			 % <---------- Partial derivative
\newcommand{\ul}{\ensuremath{\underline}} 			% <---------- underline for easier vector notation
\newcommand{\D}{\mathrm{D}}							   % <---------- Material derivate
\newcommand{\vb}{\bm {v}\xspace}						% <---------- bold,italic v
\newcommand{\wb}{\bm {w}\xspace}					  % <---------- bold,italic w
\newcommand{\ub}{\bm {u}\xspace}						% <---------- bold,italic u					
\newcommand{\gb}{\bm {g}\xspace}					   % <---------- bold,italic g
\newcommand{\rb}{\bm {r}\xspace}						 % <---------- bold,italic r
\newcommand{\fb}{\bm {f}\xspace}						 % <---------- bold,italic f
\newcommand{\Fb}{\bm {F}\xspace}						% <---------- bold,italic f
\newcommand{\ib}{\bm {i}\xspace}						  % <---------- bold,italic i
\newcommand{\jb}{\bm {j}\xspace}						  % <---------- bold,italic j
\newcommand{\yb}{\bm {y}\xspace}						% <---------- bold,italic y
\newcommand{\xb}{\bm {x}\xspace}						% <---------- bold,italic x
\newcommand{\zb}{\bm {z}\xspace}						% <---------- bold,italic z
\newcommand{\taub}{\bm {\tau}\xspace}				 % <---------- bold,italic tau
\newcommand{\lambdb}{\bm {\lambda}\xspace}
\newcommand{\til}{\ensuremath{\widetilde}} 		% <---------- tilde
\newcommand{\mub}{\bm {\mu}\xspace}				% <---------- bold, italic mu	
\newcommand{\h}{\ensuremath{\hat}} 					% <---------- hat
\newcommand{\dr}{\mathrm{d}}								% <---------- derivate


% Første side 
\title{WRF Postprocessing}
\subtitle{Dealing with the simulation data}
\author{Torgeir Blæsterdalen}
\institute{Department of Industrial Engineering, UiT-The Arctic University of Norway}
\date{\today}



% ------ Utseende og forside -----
\begin{document}
\begin{frame}
\titlepage
\end{frame}

\begin{frame}{Lecture topics}
\begin{enumerate}
	\item[1] Extracting output variables
	\item[2] Visualizing simulation data using MATLAB
\end{enumerate}
\end{frame}


\begin{frame}[fragile, allowframebreaks=1, t]{1. Extracting simulation output}
As output-files from a WRF-simulation usually is quite big, it is necessary to extract only the variables of interest from a WRF-run. A typical outfile from a domain with grid spacing of 2000 m $\times$ 2000 m produces a outfile of the magnitude 50 GB. 
A description of the variables, variable name, its units, etc. can be found by typing the command \texttt{ncdump -h <filename>}. 

When developing a MATLAB script it can be useful to download a small outfile from WRF to test on a local machine.

In MATLAB, a NetCDF-file can be inspected using the command \texttt{ncdisp}. The command for extracting a NetCDF variable is \texttt{ncread}, e.g. the line 
\begin{lstlisting}[language=MATLAB]
u  = ncread(<NetCDF-file>, 'U'); 
\end{lstlisting}
reads the variable $U$ ($x$-wind component). 

A small excursion along the surface of sphere can be expressed 
\begin{equation*}
(\delta \xb,\delta \yb, \zb) = (r\delta \lambda \cos \vartheta_o, r\delta\vartheta, z),	
\end{equation*} 
where $r$ is the radius of the sphere, $\lambda$ is the longitude, and $\vartheta$ is the latitude. The subscript $o$ denotes the an arbitrary observational point. The excursion was done at constant altitude. 

The distance from a point "$O$" to all WRF grid points can be expressed using the Pythagorean trigonometry 
\begin{equation}
d(i,j) = \Bigg(r^2\cos^2\bigg(\frac{\vartheta_o + \vartheta(i,j)}{2}\bigg)\big(\lambda_o-\lambda(i,j)\big)^2 + r^2\big(\vartheta_o - \vartheta(i,j)\big)^2 \Bigg)^{1/2}.
\label{DistanceToGridPoints}
\end{equation}
Here $r$ is the radius of the Earth, the point "$O$" is at $(\lambda_o,\vartheta_o)$ and $(\lambda(i,j),\vartheta(i,j))$ denotes the WRF grid points.

A pseudo-code for extracting an example-variable at the grid point closest to the point "$O$" calculated using Eq~\ref{DistanceToGridPoints} is given as a pseudo-code below.
\begin{lstlisting}
% Read variable from WRF output file
variable = ncread('path to WRF output file', 'variable name');

% Longitude and latitude of the site
lon = 20.6804;
lat = 69.1867;

% Find the distance to all grid points
for longitudes && latitudes in variable
d(i, j) = ... 																	(Eq.1})
end

% Find the minimum distance to the Rieppi site
[min_lon, min_lat] = find(minimum d(i,j));

save('Save variable to file')
\end{lstlisting}
\end{frame}



\begin{frame}{2. Visualizing simulation data using MATLAB}
A very useful tool for map projetions in MATLAB is the mapping functions \texttt{m\_map}. This package can be downloaded from \url{https://www.eoas.ubc.ca/~rich/map.html}.


\end{frame}




\end{document}